\documentclass[10pt,onecolumn]{witseiepaper}
\usepackage{KJN}
\usepackage{graphicx}
\usepackage{url}
\usepackage{tikz}

\newcommand{\ttt}{\texttt}

\title{ELEN4002/ELEN4012 - Minutes}
\thanks{School of Electrical \& Information Engineering, University of the
Witwatersrand, Private Bag 3, 2050, Johannesburg, South Africa}

\begin{document}

\maketitle
\pagestyle{plain}
\setcounter{page}{1}

\section*{MEETING 1}
\subsection*{Attendees:}
Kayla-Jade Butkow, Kelvin da Silva, Prof. David Rubin
\subsection*{Agenda:} 
Obtain an overview of the requirements for the project outline specification

\subsection*{Minutes of meeting:}
Start date: 09/03/2018 \\
Start time: 13:00

Requirements for the document:
\begin{enumerate}
	\item Specifications of the project
	\item Milestones: Using Gantt Chart or timeline
	\item Preliminary budget and resources 
	\item Risks or Mitigation: What do you do if you can't get a piece of equipment (what do you fall back on)
\end{enumerate}

The minimum specifications for the project are as follows: 
\begin{itemize}
	\item There must be two hearing aids (one per ear)
	\item Each hearing aid must correct for the audiogram for that ear - To do this, you tweak the filter for each ear
	\item Directionality feature comes in mainly in processing - It is filtering of a signal based on direction (can possibly be done using cross correlation). This filtering changes the signal to noise ratio 
	\item Ideas for directionality: The user changes the direction using a potentiometer or the user always hears best in the direction they are facing
	\item You must be able to turn off directionality 
	\item Testing the device using humans is not specified in the brief of the project, but it can be done 
\end{itemize}

Testing the device:
\begin{itemize}
	\item Can use a signal generator to produce pure tones (single harmonic) and then examine the resulting signal after processing 
	\item To test directionality, the sound source can be moved and the SNR examined
	\item It is also necessary to produce signals in the presence of noise and test using them
\end{itemize}

Audiogram:
\begin{itemize}
	\item An audiogram is the response of each ear to different frequencies
	\item The magnitude in dB is with reference to some standard
	\item The layout of an audiogram is given in \figref{fig:audiogram}
\end{itemize}
\begin{figure}[h]
\centering
\begin{tikzpicture}[scale=3]
\draw[->] (0,0) -- (1,0) node[right] {$Frequency (Hz)$}; 
\draw[->] (0,0) -- (0,1) node[above] {$Magnitude (dB)$};
\end{tikzpicture}
\caption{Audiogram}
\label{fig:audiogram}
\end{figure}

Budget:
\begin{itemize}
	\item Is a huge constraint in the project
	\item It doesn't include any things you have in the house such as headphones 
\end{itemize}

Ethics Clearance:
\begin{itemize}
	\item For open day, you don't need ethics clearance
	\item We need to decide how many people we would like to use to test the device
	\item We need to specify who the people are (colleagues and lecturers)
	\item Provide a protocol for testing
\end{itemize}


\section*{MEETING 2}
\subsection*{Attendees:}
Kayla-Jade Butkow, Kelvin da Silva, Prof. David Rubin
\subsection*{Agenda:} 
Review ethics clearance application

\subsection*{Minutes of meeting:}
Start date: 03/04/2018 \\
Start time: 13:00

The ethics form was reviewed and appropriate changes were made

\section*{MEETING 3}
\subsection*{Attendees:}
Kayla-Jade Butkow, Kelvin da Silva, Prof. David Rubin
\subsection*{Agenda:} 
Obtain an overview of the requirements for the project planning document

\subsection*{Minutes of meeting:}
Start date: 20/06/2018 \\
Start time: 13:00

Project Planning Document:
\begin{itemize}
	\item Needs to Gantt chart which shows how the work will be split
	\item The document is a plan of the project assuming that everything is going to go exactly according to plan 
	\item Needs to detail the full project implementation - can possibly include circuit diagrams and flow charts or algorithms
\end{itemize}

Engineering notebook:
\begin{itemize}
	\item Use it as a diary - include dates and times for all entries
	\item Write down anything that you've done on that day - eg. sketch of a circuit, phone number of a supplier
	\item Put summaries of your minutes in the engineering notebook
\end{itemize}

The aim of the project is to develop a low cost hearing aid that is very simple to use. This means that the device only has a few buttons and is not fiddly. If we can achieve this, we can possibly work with Emoyo to develop a new low cost hearing aid solution.

The optimal directionality for the project would be that you are able to tune the direction using a dial, able to hear in the direction you are facing and able to turn of the directionality. This could be implemented using a combination of directional and non-directional microphones.

People to contact for the project:
\begin{itemize}
	\item James Braid - Contacts in speech and hearing if necessary
	\item Keegan Malan - Ask for use of Kuduwave to test device and to obtain audiograms
\end{itemize}

To Do:
\begin{itemize}
	\item Remind Prof. Rubin to email us audiograms
	\item Remind Prof. Rubin to enquire about a trial version of MATLAB for the project
\end{itemize}

Going forward, meetings are to be held once a week. Prof. Rubin will be away from the 29/06/2018 until 22/07/2018 and during this time, skype meetings will be held.

\end{document}